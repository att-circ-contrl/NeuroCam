% NeuroCam manual - Development Computer
% Written by Christopher Thomas.
% Copyright (c) 2021 by Vanderbilt University. This work is released under
% the Creative Commons Attribution-ShareAlike 4.0 International License.

\chapter{Development Computer}
\label{devmachine}

The NeuroCam development environment is an x86-architecture computer running
the Linux Mint operating system, with the NeuroCam software and its required
support packages installed.

Hardware specifications are not critical, but if more than two cameras are to
be tested on a development machine, hardware performance comparable to the 
NeuroCam computer described in Chapter \ref{machine} will be needed.

To build a new NeuroCam development computer, you will need the following:
\begin{itemize}
%
\item A suitable computer.
\item A USB stick containing the Linux Mint 18.1 installer.
\item A USB stick containing the NeuroCam installer and the NeuroCam
development environment.
%
\end{itemize}

Making a NeuroCam development environment USB stick is described in Section
\ref{devmachine-usb}. Making a NeuroCAM install USB stick is described in
Section \ref{machine-usb}. Both of these may be placed on the same USB stick.

Making a Linux Mint install USB stick is described in Section
\ref{machine-usbmint}.

%
%
\section{First-Time Installation}

This is a variation of the procedure used for NeuroCam computers. There are
important differences, so be sure to check the steps closely.

\fixme{This uses the NeuroCam computer install scripts, which require the
username ``neurocam-admin''. Dedicated dev scripts would relax that
requirement.}

To install Linux and the NeuroCam software (but not the development
environment):
\begin{itemize}
%
\item Connect the machine to a keyboard, a mouse, and a monitor.
\item With the machine unpowered, plug in the Mint 18.1 USB stick.
\item Turn on the machine.
\item Get to the BIOS menu by holding the appropriate key during boot
(usually F2).
\item Turn UEFI off. This may be called ``Windows compatibility''.
\item Edit the boot order, moving the USB stick to the top.
\item Hit F10 to save and exit.
%
\item (The machine should now show the Linux boot menu.)
\item From the Linux boot menu, pick ``start in compatibility mode''.
\item From the GUI, open a terminal window.
\item Type ``\verb+sudo bash+'' to get a root session.
\item Type ``\verb+fdisk /dev/sda+'' to partition the drive.
\item Delete any existing partitions. The normally won't be any.
\item Create a 50 gigabyte partition (for the OS), an 8 gigabyte partition
(for swap space), and a third partition (for data).
\item Set the OS and data partitions to type 83 (Linux; this may be set
already by default). Set the swap partition to type 82 (Linux swap).
\item Set the ``bootable'' flag on the OS partition.
\item Save and exit \verb+fdisk+.
\item Type ``\verb+mke2fs -j /dev/sda1+'' and
``\verb+mke2fs -j -m 0 /dev/sda3+'' to create filesystems on the OS and
data partitions, respectively.
\item Type ``\verb+mkswap /dev/sda2+'' to initialize the swap partition.
\item Type ``\verb+exit+'' twice to leave the root shell and the terminal
window.
%
\item Click ``Install Linux Mint''.
\item Do not set up networks.
\item Do not install proprietary software.
\item Select ``Something Else'' for the target partition.
\item Doubleclick ``\verb+/dev/sda1+'', select ``use as ext3 journaling
filesystem'', check ``format'', and select mount point ``/''.
\item Doubleclick ``\verb+/dev/sda2+'', and select ``use as swap area''.
\item Doubleclick ``\verb+/dev/sda3+'', select ``use as ext3 journaling
filesystem'', check ``format'', and select mount point ``/data''.
\item Click ``Install Now''.
\item Select time zone and keyboard type.
\item For ``Your Name'', enter ``NeuroCam Developer''. Enter any desired
name for ``Your Computer's Name'' (allowable characters are lower--case 
letters, numbers, and hyphen; no whitespace, capitals, or punctuation).
For ``Pick a User Name'', enter ``\verb+neurocam-admin+''. For the password,
enter ``\verb+neurocam+''.
\par
\textbf{This password is easily guessed, and so should be changed when
setup is completed.} The procedure for changing passwords is described in
Section \ref{machine-password}.
\item Check ``require password to log in''.
\item Begin the install.
\item When the installation finishes, click ``restart now''.
\item Remove the USB stick when the system reboots.
%
\item (The system should boot to the Linux Mint login screen.)
\item Log in as ``\verb+neurocam-admin+'' with the password
``\verb+neurocam+''.
\par
\textbf{NOTE:} If the system does not let you log in with the credentials
supplied above, see Section \ref{machine-password} for the password reset
method.
%
\item (You should now be logged in as ``\verb+neurocam-admin+'' on a graphical
desktop.)
\item Click the terminal icon in the hotbar or start menu to get a terminal
window with a command shell.
\item In the terminal window, type ``\verb+sudo bash+'' to get a root shell.
Enter ``\verb+neurocam+'' as the \verb+sudo+ password.
\item Type ``\verb+passwd+'' to reset the \verb+root+ account's password.
Set it to ``\verb+administrator+''.
\par
\textbf{This password is easily guessed, and so should be changed when
setup is completed.} The procedure for changing passwords is described in
Section \ref{machine-password}.
\item Type ``\verb+exit+'' to leave the root shell.
\item Type ``\verb+exit+'' again to leave the command shell, closing the
terminal window.
\item Click on the ``log out'' icon in the start menu.
%
\item Press \verb+ctrl-alt-F6+ to get to a text console (if using a Mac
keyboard, \verb+ctrl-option-function-F6+).
\item Log in as ``\verb+root+'' with the password ``\verb+administrator+''.
\item Plug the development computer into an internet jack.
\item Wait 10 seconds.
\item Type ``\verb+ifconfig+'' to get network interface information. Wired
ethernet is the entry with a device name starting with ``\verb+e+'' (usually
``\verb+eth+'', ``\verb+eno+'', ``\verb+enp+'', or similar).
\item When network handshaking has finished, there will be an
``\verb+inet addr+'' field with an IP address assigned. This address will
\textbf{not} start with \verb+127+ (\verb+127.x.x.x+ is the loopback address).
\item Your institution's network may require network cards' MAC addresses
to be registered before allowing connections. To find the network card's MAC
address, look for a field named ``\verb+HWaddr+''. Write this down.
\item Type ``\verb+ping 8.8.8.8+'' to check internet connectivity. A response
of ``\verb+64 bytes from 8.8.8.8+'' means that the internet is visible.
\item Type ``\verb+mkdir /usb+'' to create a manual mount point for the USB
stick. This only needs to be done once.
\item Make sure no other USB sticks are in the machine, and insert a
NeuroCam update USB stick.
\item Type ``\verb+mount -t auto -o exec /dev/sdb1 /usb+'' to manually mount
the USB stick and to allow scripts to be run from the stick.
\item Type ``\verb+/usb/neurocam-install/scripts/do-install.sh+'' to perform
first-time NeuroCam software installation. This will take a while.
\par
\textbf{NOTE:} This should skip most confirmation steps, but may still ask for
user input. Default settings should always be acceptable.
\item Once this has finished, type ``\verb+shutdown -r now+'' to reboot.
\item Remove the USB stick when the system reboots.
%
\end{itemize}

To install the development environment:
\begin{itemize}
%
\item Allow the machine to boot to the graphical login screen as normal.
\par
(To switch from the text logon screen to the graphical screen, press
\verb+ctrl-alt-F8+; on a Mac keyboard, \verb+ctrl-option-function-F8+.)
\item Log in as ``\verb+neurocam-admin+''.
\item Make sure no other USB sticks are in the machine, and insert a NeuroCam
development USB stick. Close the file browser popup window, if any.
\item Click the terminal icon in the hotbar or start menu to get a terminal
window with a command shell.
\item Navigate to the directory you wish to use as the development tree root.
You can create a new directory with ``\verb+mkdir ~/(directory)+'' and move
to it with ``\verb+cd ~/(directory)+''. Type ``\verb+pwd+'' to check that you
are in the desired location.
\item Type ``\verb+tar -xvf /media/neurocam-admin/(label)/neurocam/dev/*.tar+''.
\par
This refers Mint's automatic mount point for the USB stick, with 
``\verb+(label)+'' replaced with the stick's volume label (or serial number if
there is no volume label). You can use ``tab completion'' to avoid having to
type this: if only one USB stick is plugged in, 
``\verb+/media/neurocam-admin/(tab)+'' will automatically expand to the
correct mount point name when the \verb+tab+ key is pressed.
\item When the ``\verb+tar+'' command has finished extracting files, type
``\verb+ls+'' to check that the development tree's subdirectories are in 
place.
\item Type ``\verb+umount /media/neurocam-admin/(label)+'' to unmount the
USB stick for safe removal.
%
\end{itemize}

%
%
\section{Updating the Development Machine}

Linux will usually update itself automatically, but doing this manually is
also acceptable. The NeuroCam software and the NeuroCam development code
will have to be updated manually.

Updating Linux Mint requires an internet connection. Updating the NeuroCam
software and development code do not.

To update Linux Mint:
\begin{itemize}
%
\item (Turn on the machine and allow it to boot per normal.)
\item Press \verb+ctrl-alt-F6+ to get to a text console.
\item Log in as ``\verb+root+''.
\item Type ``\verb+ping 8.8.8.8+'' to check internet connectivity. A response
of ``\verb+64 bytes from 8.8.8.8+'' means that the internet is visible.
\item Type ``\verb+~/neurocam-scripts/do-mintupdate.sh+''. This may take a
while, depending on how many packages need to be updated.
\par
\textbf{NOTE:} This should skip most confirmation steps, but may still ask for
user input. Default settings should always be acceptable.
\item Type ``\verb+exit+'' to log out.
\item Press \verb+ctrl-alt-F8+ to return to the graphical login screen.
\end{itemize}

To update the NeuroCam software:
\begin{itemize}
%
\item (Turn on the machine and allow it to boot per normal.)
\item Press \verb+ctrl-alt-F6+ to get to a text console.
\item Log in as ``\verb+root+''.
\item Make sure no other USB sticks are in the machine, and insert a
NeuroCam update USB stick.
\item Wait five seconds, then type ``\verb+mount /usb+''.
\item Type ``\verb+~/neurocam-scripts/do-update.sh+''.
\item Once this has finished, type ``\verb+shutdown -r now+'' to reboot.
This will force a full restart of the NeuroCam software.
\item Remove the USB stick when the system reboots.
%
\end{itemize}

To update the NeuroCam development environment:
\begin{itemize}
\item (Turn on the machine and allow it to boot to the graphical login
screen as normal.)
\item Log in as ``\verb+neurocam-admin+''.
\item Make sure no other USB sticks are in the machine, and insert a NeuroCam
development USB stick. Close the file browser popup window, if any.
\item Click the terminal icon in the hotbar or start menu to get a terminal
window with a command shell.
\item Navigate to the directory you use as the development tree root.
Type ``\verb+pwd+'' to check that you are in the desired location.
\item \textbf{Make sure that you have backed up any changed files.}
Reinstalling the development tree will overwrite any existing files that are
present.
\item Type ``\verb+tar -xvf /media/neurocam-admin/(label)/neurocam/dev/*.tar+''.
\item When the ``\verb+tar+'' command has finished extracting files, type
``\verb+ls+'' to check that the development tree's subdirectories are in 
place.
\item Type ``\verb+umount /media/neurocam-admin/(label)+'' to unmount the
USB stick for safe removal.
\end{itemize}

%
%
\section{Making New Development USB Sticks}
\label{devmachine-usb}

The NeuroCam development environment can be added to any USB stick. This
does not interfere with existing data; the software is placed in a new
directory called ``\verb+neurocam-dev+''.

To add the install and update software to an already-formatted USB stick:
\begin{itemize}
\item On a NeuroCam development machine, open a terminal window.
\item Navigate to the NeuroCam development directory.
\item Make sure no other USB sticks are in the machine, and insert a USB
stick to turn into a NeuroCam update USB stick.
\item Type ``\verb+install/scripts/make-devkey.sh+''.
\item Wait until the script has finished.
\item Type ``\verb+umount /media/neurocam-admin/(label)+'' to unmount the
USB stick for safe removal.
\end{itemize}

%
% This is the end of the file.
