% NeuroCam manual - NeuroCam Computer
% Written by Christopher Thomas.
% Copyright (c) 2021 by Vanderbilt University. This work is released under
% the Creative Commons Attribution-ShareAlike 4.0 International License.

\chapter{NeuroCam Computer}
\label{machine}

The NeuroCam computer is a small--form--factor x86--architecture computer
running the Linux Mint operating system and the NeuroCam software.

To build a new NeuroCam computer, you will need the following:
\begin{itemize}
%
\item A suitable computer.
\item A USB stick containing the Linux Mint installer.
\item A USB stick containing the NeuroCam installer.
%
\end{itemize}

Details of the hardware and the installation process are described in their
appropriate sections. The procedure for making new USB sticks is also
described.

%
%
%
\section{Hardware}

The NeuroCam computer must be powerful enough to perform image compositing
in real-time, have a solid--state drive that is fast, large, and has high
endurance, and have enough RAM to handle any cacheing and buffering
transparently.

The components used for the NeuroCam prototype are as follows:

\begin{tabular}{llll}\hline
Qty & Description & Manuf. p/n & NewEgg SKU \\
\hline
%
1 & Intel NUC with Core i7 & Intel NUC6i7KYK & N82E16856102166 \\
% This is 1x 8 gig.
1 & DDR4 260-pin SO-DIMM 8G$^*$ & Ripjaws F4-2133C15S-8GRS & N82E16820232147 \\
% This is 2x 4 gig. We only _need_ 1x 4 gig.
%1 & DDR4 260-pin SO-DIMM 4G$^*$ & Ripjaws F4-2133C15D-8GRS & N82E16820232146 \\
1 & SSD 1 TB high-endurance M.2 & Samsung MZ-N5E1T0BW & N82E16820147567 \\
%
\hline
\multicolumn{4}{l}{$^*$A single 4-gig stick is sufficient, but no longer in
NewEgg's catalogue.} \\
\end{tabular}

To assemble the computer (Intel NUC version):
\begin{itemize}
%
\item Ensure that the workspace is free of clutter and clean.

\item Ensure that clothing is not carrying electrostatic charge. A humidifier
can reduce static electricity in the workspace if necessary.

\item Remove the skull-logo faceplate and set it aside. Unpack the plain
faceplate as a replacement.

\item With the machine interior exposed, carefully seat the RAM in the lower
DIMM slot. Apply \textbf{gentle} pressure until the latches engage to secure
the RAM. It may be necessary to open the latches by hand to fully insert the
RAM.

\item Remove the securing screw for the first solid-state drive slot, and
insert the solid-state drive. Ensure that the drive is seated firmly in its
connector, and reinstall the securing screw.

\item Reinstall the faceplate.
%
\end{itemize}

%
%
%
\section{Installing Mint and the NeuroCam Software}

The prototype NeuroCam machine used Linux Mint 18.1. This version or any later
version should have full driver support for the specific Intel NUC machine
described above.

You will need a Mint 18.1 install USB stick and a NeuroCam install USB stick.
Making these is described in Sections \ref{machine-usbmint} and
\ref{machine-usb}, respectively.

%
\subsection{First-Time Installation}

To install Linux and the NeuroCam software:
\begin{itemize}
%
\item Connect the machine to a keyboard, a mouse, and an HDMI monitor.
\item With the machine unpowered, plug in the Mint 18.1 USB stick.
\item Turn on the machine.
\item Hit F2 to get to the BIOS menu.
\item Turn UEFI off. This may be called ``Windows compatibility''.
\item Edit the boot order, moving the USB stick to the top.
\item Hit F10 to save and exit.
%
\item (The machine should now show the Linux boot menu.)
\item From the Linux boot menu, pick ``start in compatibility mode''.
\item From the GUI, open a terminal window.
\item Type ``\verb+sudo bash+'' to get a root session.
\item Type ``\verb+fdisk /dev/sda+'' to partition the solid--state drive.
\item Delete any existing partitions. The normally won't be any.
\item Create a 50 gigabyte partition (for the OS), an 8 gigabyte partition
(for swap space), and a third partition (for data).
\item Set the OS and data partitions to type 83 (Linux; this may be set
already by default). Set the swap partition to type 82 (Linux swap).
\item Set the ``bootable'' flag on the OS partition.
\item Save and exit \verb+fdisk+.
\item Type ``\verb+mke2fs -j /dev/sda1+'' and
``\verb+mke2fs -j -m 0 /dev/sda3+'' to create filesystems on the OS and
data partitions, respectively.
\item Type ``\verb+mkswap /dev/sda2+'' to initialize the swap partition.
\item Type ``\verb+exit+'' twice to leave the root shell and the terminal
window.
%
\item Click ``Install Linux Mint''.
\item Do not set up networks.
\item Do not install proprietary software.
\item Select ``Something Else'' for the target partition.
\item Doubleclick ``\verb+/dev/sda1+'', select ``use as ext3 journaling
filesystem'', check ``format'', and select mount point ``/''.
\item Doubleclick ``\verb+/dev/sda2+'', and select ``use as swap area''.
\item Doubleclick ``\verb+/dev/sda3+'', select ``use as ext3 journaling
filesystem'', check ``format'', and select mount point ``/data''.
\item Click ``Install Now''.
\item Select time zone and keyboard type.
\item For ``Your Name'', enter ``NeuroCam User''. For ``Your Computer's
Name'', enter ``\verb+neurocam-NN+'', where ``\verb+NN+'' is the number
assigned to this NeuroCam machine. For ``Pick a User Name'', enter
``\verb+neurocam-admin+''. For the password, enter ``\verb+neurocam+''.
\par
\textbf{This password is easily guessed, and so should be changed when the
system is installed per Chapter \ref{setup}.}
\item Check ``require password to log in''.
\item Begin the install.
\item When the installation finishes, click ``restart now''.
\item Remove the USB stick when the system reboots.
%
\item When booting to a new install, press \verb+ctrl-alt-F6+ to switch to
text console \#6 (1 through 6 are valid). If using a Mac keyboard, use
\verb+ctrl-option-function-F6+.
\item Log in as ``\verb+neurocam-admin+'' with the password
``\verb+neurocam+''.
\par
\textbf{NOTE:} If the system does not let you log in with the credentials
supplied above, see below for the password reset method.
%
\item (You should now be logged in as ``\verb+neurocam-admin+'' on a text
console.)
\item Type ``\verb+sudo bash+'' to get a root shell. Enter
``\verb+neurocam+'' as the \verb+sudo+ password.
\item Type ``\verb+passwd+'' to reset the \verb+root+ account's password.
Set it to ``\verb+administrator+''.
\par
\textbf{This password is easily guessed, and so should be changed when the
system is installed per Chapter \ref{setup}.}
\item Type ``\verb+exit+'' twice to leave the root shell and the console 
login session.
\item (You should now be at a console login prompt.)
\item Log in as ``\verb+root+'' with the password ``\verb+administrator+''.
\item Type ``\verb+nano /etc/default/grub+'' to edit the bootloader
configuration file.
\item Press \verb+ctrl-w+ to search, and search for ``\verb+quiet splash+''.
\item Change ``\verb+quiet splash+'' to ``\verb+quiet nosplash text+''.
\item Press \verb+ctrl-o+ to save, and \verb+ctrl-x+ to exit.
\item Type ``\verb+update-grub2+'' to apply the configuration change.
\item Type ``\verb+systemctl disable mdm+'' to turn off the GUI manager.
\item Type ``\verb+shutdown -r now+'' to reboot.
%
\item (The machine should boot per normal.)
\item If the machine has a black screen, press \verb+ctrl-alt-F6+ to get
to a text console.
\item Log in as ``\verb+root+'' with the password ``\verb+administrator+''.
\item Type ``\verb+ifconfig+'' to get network interface information. Look
for a field named ``\verb+HWaddr+''; this is the MAC address for a given
network interface. Write down (and doublecheck) the MAC address for the
ethernet jack (the device name will start with ``\verb+eno+'' or
``\verb+eth+'').
\item Plug the NeuroCam computer into one of the router's LAN ports.
\item Add the hardware address to the router's whitelist so that the
NeuroCam computer can see the network (per Chapter \ref{router}).
\item Wait 10 seconds, and then type ``\verb+ifconfig+'' again. When network
handshaking has finished, there will be an ``\verb+inet addr+'' field with
an IP address assigned. This address should be ``\verb+192.168.1.NN+'', for
some number ``\verb+NN+''.
\item Plug an internet cable into the router's WAN port so that the internet
is visible.
\par
\textbf{This is needed in order to update the operating system, but should
otherwise be disconnected.}
\item Type ``\verb+ping 8.8.8.8+'' to check internet connectivity. A response
of ``\verb+64 bytes from 8.8.8.8+'' means that the internet is visible.
\item Type ``\verb+mkdir /usb+'' to create a manual mount point for the USB
stick. This only needs to be done once.
\item Make sure no other USB sticks are in the machine, and insert a
NeuroCam update USB stick.
\item Type ``\verb+mount -t auto -o exec /dev/sdb1 /usb+'' to manually mount
the USB stick and to allow scripts to be run from the stick.
\par
\textbf{NOTE:} If a second solid--state drive is in the system (see below),
use ``\verb+/dev/sdc1+'' instead of ``\verb+/dev/sdb1+'' above.
\item Type ``\verb+/usb/neurocam-install/scripts/do-install.sh+'' to perform
first-time NeuroCam software installation. This will take a while.
\par
\textbf{NOTE:} This should skip most confirmation steps, but may still ask for
user input. Default settings should always be acceptable.
\item Once this has finished, type ``\verb+shutdown -r now+'' to reboot.
\item Remove the USB stick when the system reboots.
\item Disconnect the router from the internet by unplugging the WAN cable.
%
\end{itemize}

%
\subsection{Updating Linux and the NeuroCam Software}

Updating Linux may be done whenever desired. This normally isn't needed,
unless a NeuroCam software update indicates that it needs updated OS
packages as well.

\textbf{NOTE:} If the NeuroCam computer is ever exposed to the internet or
to any other external network, keeping Linux updated is a good idea, as this
will patch security holes that are discovered in its software.

Updating Linux Mint requires an internet connection. Updating the NeuroCam
software does not.

To update Linux Mint:
\begin{itemize}
%
\item (Turn on the machine and allow it to boot per normal.)
\item If the machine has a black screen, press \verb+ctrl-alt-F6+ to get
to a text console.
\item Log in as ``\verb+root+'' with the password ``\verb+administrator+''.
\item Plug the NeuroCam computer into one of the router's LAN ports.
\item Wait 10 seconds, and then type ``\verb+ifconfig+''. When network
handshaking has finished, there will be an ``\verb+inet addr+'' field with
an IP address assigned. This address should be ``\verb+192.168.1.NN+'', for
some number ``\verb+NN+''.
\item Plug an internet cable into the router's WAN port so that the internet
is visible.
\par
\textbf{This is needed in order to update the operating system, but should
otherwise be disconnected.}
\item Type ``\verb+ping 8.8.8.8+'' to check internet connectivity. A response
of ``\verb+64 bytes from 8.8.8.8+'' means that the internet is visible.
\item Type ``\verb+~/neurocam-scripts/do-mintupdate.sh+''. This may take a
while, depending on how many packages need to be updated.
\par
\textbf{NOTE:} This should skip most confirmation steps, but may still ask for
user input. Default settings should always be acceptable.
\item Once this has finished, type ``\verb+shutdown -r now+'' to reboot.
\item Disconnect the router from the internet by unplugging the WAN cable.
\end{itemize}

To update the NeuroCam software:
\begin{itemize}
%
\item (Turn on the machine and allow it to boot per normal.)
\item If the machine has a black screen, press \verb+ctrl-alt-F6+ to get
to a text console.
\item Log in as ``\verb+root+'' with the password ``\verb+administrator+''.
\item Make sure no other USB sticks are in the machine, and insert a
NeuroCam update USB stick.
\item Wait five seconds, then type ``\verb+mount /usb+''.
\item Type ``\verb+~/neurocam-scripts/do-update.sh+''.
\item Once this has finished, type ``\verb+shutdown -r now+'' to reboot.
\item Remove the USB stick when the system reboots.
%
\end{itemize}

%
\subsection{Logging In Over the Network}

When logging into a NeuroCam machine on--site, installing a monitor might not 
be practical. As long as a machine is available that is authorized to connect
to the NeuroCam network, logging in can be done remotely.

To log into the NeuroCam machine using a network connection:
\begin{itemize}
\item Connect to the NeuroCam system's wireless network.
\item Connect to the NeuroCam machine using the ``\verb+ssh+'' protocol with
username ``\verb+neurocam-admin+''. Under Linux or MacOS, this can be done
from a terminal window by typing ``\verb+ssh neurocam-admin@192.168.1.NN+'',
where ``\verb+NN+'' is from the IP address from the sticker on the NeuroCam 
machine. Under Windows, a ``terminal program'' such as ``\verb+PuTTY+'' may
be needed.
\item Enter the ``\verb+neurocam-admin+'' account's password when prompted.
\item Type ``\verb+su+'' (\textbf{not} ``\verb+sudo+'').
\item Enter the ``\verb+root+'' account's password when prompted.
\item Type ``\verb+cd ~+''
\item You are now logged in as ``\verb+root+'' and are in \verb+root+'s
home directory.
\end{itemize}

%
\subsection{Resetting Passwords}
\label{machine-password}

To reset a password for an account that you know the existing password for:
\begin{itemize}
\item Log in as that account.
\item Type ``\verb+passwd+''.
\item Enter the new password when prompted.
\end{itemize}

To reset the ``\verb+neurocam-admin+'' password when you \textit{can} log
in as ``\verb+root+'':
\begin{itemize}
\item Log in as ``\verb+root+''.
\item Type ``\verb+passwd neurocam-admin+'' to reset the account password
for ``\verb+neurocam-admin+''.
\item Enter the new password when prompted.
\end{itemize}

To reset the ``\verb+root+'' password when you \textit{can} log in as
``\verb+neurocam-admin+'':
\begin{itemize}
\item Log in as ``\verb+neurocam-admin+''.
\item Type ``\verb+sudo bash+'' to get a root shell. Enter the password
for the ``\verb+neurocam-admin+'' account as the \verb+sudo+ password.
\item Type ``\verb+passwd+''.
\item Enter the new password when prompted.
\item Type ``\verb+exit+'' to leave the root shell.
\end{itemize}

To reset the ``\verb+neurocam-admin+'' password when you \textit{cannot}
log into the NeuroCam machine at all, do the following:
\begin{itemize}
\item With the machine unpowered, plug in the Mint 18.1 USB stick.
\item Turn on the machine.
\item Hit F10 to enter the boot menu.
\item Select the USB stick from the boot devices, and boot.
\item (The machine should now show the Linux boot menu.)
\item From the Linux boot menu, pick ``start in compatibility mode''.
\item From the GUI, open a terminal window.
\item Type ``\verb+sudo bash+'' to get a root session.
\item Type ``\verb+mount -t auto /dev/sda1 /mnt+'' to mount the hard disk
in the ``\verb+/mnt+'' mount point.
\item Type ``\verb+chroot /mnt+'' to open a new shell that uses
``\verb+/mnt+'' as the root folder.
\item Type ``\verb+passwd neurocam-admin+'' to reset the account password
for ``\verb+neurocam-admin+''.
\item Enter the new password when prompted.
\item Type ``\verb+shutdown -r now+'' to reboot.
\item Remove the USB stick when the system reboots.
\end{itemize}

%
\subsection{Adding a Second Drive}

The instructions above configure a machine to use a single solid--state drive.
A second drive may be added, and given a single data partition; the two data
partitions (on the first and second drive) may then be configured as a single
larger \verb+RAID0+ drive.

\fixme{This hasn't been implemented, so no documentation for it.}

\fixme{Cover ``doing this before install'' and ``modifying a machine after
install'' separately.}

%
%
\section{Making New NeuroCam Install USB Sticks}
\label{machine-usb}

The NeuroCam install and update software can be added to any USB stick. This
does not interfere with existing data; the software is placed in a new
directory called ``\verb+neurocam-install+''.

To add the install and update software to an already-formatted USB stick:
\begin{itemize}
\item On a NeuroCam development machine, open a terminal window.
\item Navigate to the NeuroCam development directory.
\item Make sure no other USB sticks are in the machine, and insert a USB
stick to turn into a NeuroCam update USB stick.
\item Type ``\verb+install/scripts/make-installkey.sh+''.
\item Wait until the script has finished.
\item Type ``\verb+umount /media/neurocam-admin/(label)+'' to unmount the
USB stick for safe removal.
\par
This refers to Mint's automatic mount point for the USB stick, with 
``\verb+(label)+'' replaced with the stick's volume label (or serial number if
there is no volume label). You can use ``tab completion'' to avoid having to
type this: if only one USB stick is plugged in,
``\verb+/media/neurocam-admin/(tab)+'' will automatically expand to the
correct mount point name when the \verb+tab+ key is pressed.
\end{itemize}

%
%
\section{Making New Linux Mint USB Sticks}
\label{machine-usbmint}

The Linux Mint install software requires a dedicated USB stick. Adding the
Linux Mint installer destroys all other contents of the stick.

The version of Linux Mint used by the NeuroCam machines as of this writing
is 18.1.

To make a Linux Mint USB stick:
\begin{itemize}
\item On a Linux machine (such as a NeuroCam development machine), open
a browser and go to ``\verb+https://www.linuxmint.com+''.
\item Click on the ``Download'' tab.
\item Check the version number shown on the download page. If this does not
match the desired version, click on the ``All versions'' tab, and select the
desired version.
\item Select the 64-bit version with your desired desktop. NeuroCam
development was done using the ``Cinnamon'' desktop, but other desktops
should work.
\item Choose a mirror in the appropriate country to download the \verb+.iso+
image for your selected distribution. This may take some time to download.
Save this image and make note of its name and where you put it.
\item Click the terminal icon in the hotbar or start menu to get a terminal
window.
\item Type ``\verb+sudo apt-get install unetbootin+'' to make sure the
boot stick creation application is present. Enter your password when
prompted.
\item Type ``\verb+cat /proc/partitions+''. Insert the USB stick, close the
file browser popup (if any), then type ``\verb+cat /proc/partitions+'' again.
The newly-added lines indicate the device name of the USB stick (usually
``\verb+/dev/sdb+'' for the stick itself and ``\verb+/dev/sdb1+'' for the
data partition on it).
\item If desired, reformat the USB stick and set a meaningful volume label.
\par
To do this manually, ``eject'' (unmount) the USB stick, and type \\
``\verb+sudo mke2fs -j -m 0 -L (label) /dev/(partition device)+''. \\
Enter your password when prompted.
\item Type
``\verb+unetbootin method=diskimage isofile=(file) installtype=USB+ \\
\verb+targetdrive=/dev/(partition)+''.
\item Enter your password when prompted.
\item The UNetbootin dialog should already have the ``Diskimage'' method
selected, of type ``ISO'', with the filename filled in. Type should already
be ``USB Drive'', with target drive set to the partition name on the USB
stick. Verify this information, and click ``Ok''.
\item (UNetbootin will copy the install image to the USB drive and install a
boot loader.)
\item Click ``Exit'' to exit UNetbootin.
\item Type ``sync'' to commit all changes to disk.
\item Eject the USB drive, remove it, and then verify it by attempting to
boot a test machine using it.
\end{itemize}

%
% This is the end of the file.
